\chapter{Stopped flow \SOtwo measurements}\label{ch:stoppedflow}

As a first step in this research, experiments on blood using a stopped flow setup were completed, using a similar protocol and design used in previous work by the group.
These experiments used the stopped flow experimental setup from \autoref{ch:exptsetup}, with 3 different magnet systems at different fields to observe changes in \Ttwo at a number of levels of oxygenation.

\section{Experimental Protocol}
For this series of experiments, measurements of \Ttwo were made at a series of oxygenation steps, which we attempted to set by adjusting the gas mix going into the oxygenator.
These steps are shown graphically in TODO fig from ICMRM!!
After the flow circuit was assembled, the blood sample was loaded into the tubes and oxygenator, using the standard connectors on the bag.
The pump was switched on, and gas at 21\% \Otwo was flowed through to bring the oxygenation and temperature of the blood up to the 100\% starting point for the experiments.
The blood was allowed to flow through the circuit for TODO how long?? to stabilise, before an initial test with the iStat.
Durig this time, the probes were tuned and matched, and pulse sequence parameters such as pulse power were calibrated using amplitude sweeps (using FID on the Halbach system, and CPMG detection on the MOLE.)

If this showed that the pH, \SOtwo and HcT were normal, the flow was stopped and measurements on the two in-line NMR systems were started.
Typically \TR = \SI{1.5}{s}, with echo times of \SIlist{0.25;0.5;1;5}{ms} on the Halbach, and \SIlist{0.25; 0.5; 1}{ms} on the MOLE (these were limited by the magnetic field gradients in each system).
4 scans with phase cycling were used on the Halbach, but the poorer signal to noise on the MOLE meant that 16 scans were required.
Once these measurements had completed, the pump was turned on briefly to agitate the blood and minimize any effects of settling before repeating the NMR measurements.
A \SI{2}{ml} sample of blood in the circuit was also taken out and syringed into a clean \SI{5}{mm} NMR tube for the Spinsolve measurements.
These tubes were sealed with the caps provided by the manufacturer.
On the Spinsolve, typical experimental parameters were \TR = \SI{10}{s} with a 4 scan phase cycle, and with 4 echo times: \SIlist{0.5;2.5;5;10}{ms}.

The \SOtwo was then lowered to an intermediate value (around 50\%).
Flow was turned back on and the gas mix set to 5\% \Otwo, which we had found produced these levels of oxygenation.
After waiting for the \SOtwo to reach the desired level, a sample of blood was taken from the flow circuit and the iStat was used to confirm the \SOtwo.
The flow was stopped, and two sets of NMR meaurements were completed as above, with samples also removed for the Spinsolve.

This process was repeated again to measure \Ttwo at a very low oxygenation (10\% or less), by setting the gas mix to 0\% \Otwo.
Afterwards, these steps were reversed to increase the oxygenation again, taking a measurement at an intermediate value of \SOtwo on the way up as well.

Data was extracted and processed using Python, in JuPyter notebooks.
Each echo train is phased, and each echo is summed to give a single value for the signal at each echo time.
The resulting decay is fit to a monoexponential decay, which results in a single \Ttwo value at each \SOtwo.

\section{Results}

Short section of results presented at ICMRM - 3 sO2 at 3 different fields


\section{Discussions}

\begin{itemize}
\item Hard to set sO2 with gas mixer
\item Effect of blood settling when flow stopped
\item time taken to measure sO2 with iStat
\end{itemize}
