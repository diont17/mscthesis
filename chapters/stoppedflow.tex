\chapter{Stopped flow \SOtwo measurements}\label{ch:stoppedflow}

As a first step in this research, experiments on blood using a stopped flow setup were completed, using a similar protocol and design used in previous work by the group.
These experiments used a flow circuit and 3 different magnet systems with different fields to observe changes in \Ttwo at a number of levels of oxygenation.

Results from these experiments were presented at the 14\textsuperscript{th} International Congress on Magnetic Resonance Microscopy.

\section{Experimental Setup}

\subsection{Flow circuit}
The blood used in these experiments was contained in a flow circuit similar to what is used for Cardio-Pulmonary Bypass (CPB) surgery.
A schematic of the flow setup is shown in TODO drawing.
The main components are the roller pump and the oxygenation membrane, with the blood collection bag acting as a reservoir.
These are all joined by 1/4'' medical grade PVC tubing.
The total volume of this circuit was typically 60 ml, with the blood bag containing 450 ml TODO check logbook.
The roller pump is used to generate flow around the circuit, by squeezing the walls of the tube and forcing the blood along.
The pump is also designed for use in CPB surgery, and is a Stockert SIII, with a flow rate controllable from 0.01 to 6 L/min when used with the 1/4 '' tube. TODO equip?.
These versions of the flow circuit included a bypass section controlled by clamping the tubes so that the blood flow could be directed to around the reservoir, allowing for faster changes in the oxygenation due to a lower effective volume.


\subsection{Oxygenator and Gas Mixer}
A hollow-fibre membrane oxygenator was used to control the oxygenation of the blood in the circuit.
This was a Medtronic Affinity Pixie oxygenator, which is designed for use in paediatric CPB procedures.
It allows for gas exchange and temperature control of the blood flowing through, with the oxygenation controlled by the mix of gases going into the oxygenator.
This paediatric model was chosen to minimize the required priming volume, and it still provided more than enough capacity to oxygenate the blood at the flow rates used.
The temperature of the blood was regulated by flowing water from a temperature controlled waterbath through the oxygenator, which has a separate path built into it for this purpose.
The water bath temperature was set to 35 degrees Celsius TODO symbols?, although this corresponded with an outlet temperature of 34 C at maximum flow rates (with lower temperatures at lower flow rates.)

The gas mix was set using a Dansensor MapMix 3 gas mixer, which allowed control of the different proportions of N2, O2 and CO2 flowing through the oxygenator.
The Oxygen fraction was typically varied from 0\% up to 21\%, the value of atmospheric air.
Nitrogen was used as a non-Oxygen containing gas, while between 3-5\% CO2 was also added to ensure the pH of the blood sample remained stable over the course of the experiment (as these are linked by the bicarbonate buffer system in blood).
Gases for all experiments were used as supplied from BOC and were instrument grade or higher.
The output of the gas mixer flowed into a pressurised buffer tank before flowing into a Dansensor MapCheck Provectus gas analyser, which allowed us to check the fraction of the three gases which flowed into the oxygenation membrane.
This gas analyser also provided some flow regulation, although this was supplemented with an adjustable needle valve.
Typically, the gas flow rate was set to 1-2 Lpm to avoid creating bubbles in the oxygenator.

\subsection{Blood collection}
Samples of whole blood were collected by venipuncture from healthy volunteers (taking place at Otago medical school).
This procedure was approved by the central health and disability ethics committee, and all volunteers provided written, informed consent.
Approximately 450 ml of blood was collected into a blood bag containing 66.5 ml CPD anticoagulant/preservative solution (Haemonetics Leukotrap WB system).
CPD solution is rated for storage of red blood cells for up to at least 3 weeks of storage. TODO cite this
Blood samples were held at room temperature following collection, before undergoing leukoreduction filtering to remove white blood cells.
It has been shown that the presence of white blood cells can adversely affect the condition of red blood cells in storage.
Samples were then moved to a refrigerator and stored at 4 degrees Celsius until required for experiments (up to 30 TODO check! days.)

\subsection{NMR setup}
For these experiments, 3 different permanent magnet systems were used to obtain data at 3 different field strengths.
The first system was a 12 MHz (0.3 T) Halbach magnet array, with 9 cm diameter bore and approximately 20 cm long.

This was combined with the


\begin{itemize}
\item Flow Circuit
\item Oxygenator, gas mixer
\item Blood collection and age?
\item NMR setup
\item Coil and electronics
\end{itemize}

\section{Results}
Short section of results presented at ICMRM - 3 sO2 at 3 different fields
\section{Discussions}

\begin{itemize}
\item Hard to set sO2 with gas mixer
\item Effect of blood settling when flow stopped
\item time taken to measure sO2 with iStat
\end{itemize}
