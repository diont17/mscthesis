\chapter{Models for \Ttwo relaxation enhancement}\label{ch:models}

As discussed in Chapter \ref{ch:background}, researchers have developed a number of models to explain the decrease in \Ttwo of the water protons in blood.
Thulborn originally applied the Luz-Meiboom equation to describe the decrease in \Ttwo and its dependence on the CPMG echo time\cite{ThulbornOxygenationdependencetransverse1982}.
This assumes that the water protons in blood instantaneously exchange between two sites with different Larmor frequencies, with a rate $\frac{1}{\tau_{ex}}$.
While this provides good agreement with experimental data, it does not provide a good description of the underlying physical system - for example, the exchange time in found in previous studies do not agree with measurements of the exchange rate across the red blood cell membrane \cite{MeyerNMRrelaxationrates1995}.
There are also inconsistencies in the values of $\tau_{ex}$ reported in the literature, with values ranging from 0.3 ms to 10.1 ms (see table 2).
Another model has been proposed by Jensen and Chandra, which more accurately describes the dephasing of water protons that diffuse through areas of magnetic field inhomogeneities\cite{JensenNMRrelaxationtissues2000}.
Generally, the two models predict the same behaviour at short echo times (<3 ms) and at long echo times (>20ms)\cite{BrooksT2shorteningweaklymagnetized2001}, but in the intemediate range, the two models give different values for the magnitude of the \Ttwo change.
Additionally, experiments comparing the two models have found that at clinical MRI fields (1.5 T and up), both models provide good agreement with experimental data \cite{StefanovicHumanwholebloodrelaxometry2004,GardenerDependencebloodR22010,GrgacHematocritoxygenationdependence2013}.
Because of this, the Luz-Meiboom equation is typically used when analysing data, as it is simpler than the diffusion model.

As another component of this research, the agreement between these models and data collected at lower fields has been tested.

\section{Theory}

\section{Experimental Setup}

\subsection*{Analysis method}
this
\section{Results}

\section{Discussion}
