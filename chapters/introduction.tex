\chapter{Introduction}\label{ch:intro}

Blood oxygenation is a critical parameter for assessing patient health and the management of disease.
Oxygen saturation measurement are a routine part of care, and changes in these measurements can act as warning signs of acute/serious problems in the body.
Low levels of oxygen can cause tissue damage and cell death, particularly in sensitive tissue such as the brain, where even brief periods of hypoxia can have catastrophic effects.
Low oxygen levels can also be an issue for neonates, who can have circulatory and pulmonary systems which are still developing.
Being able to quickly detect changes in oxygenation means that action can be taken to limit and/or prevent ill effects to patients.

The typical method for measuring blood oxygenation in patients is through pulse oximetry, a non-invasive method which measures the change in absorbance at various wavelengths of light to detect the presence of oxygenated and deoxygenated haemoglobin.
This method can produce quick measurements of the blood oxygen saturation, while also being relatively cheap and robust.
As it measures the blood, this technique provides systemic information about Oxygen saturation, measured at a single point on the finger or ear lobe.
However, these blood saturation values are not necessarily representative of Oxygen levels in specific parts of the body.
Problems with blood flow (ischemia) to sensitive areas in the body, can cause significant damage, and do not necessarily show up on pulse oximetry.
Because of this, physicians desire local tissue oxygenation measurements, but they are difficult to measure.

Techniques for measuring local tissue oxygenation in the brain include the use of NIRS (Near-Infrared Spectroscopy), and the use of oxygen measuring
polarographic electrodes.
NIRS is a non-invasive method where the changes in light absorption from oxy/deoxy-haemoglobin are observed at near-infrared wavelengths, which can penetrate
into tissue, similar to pulse oximetry.
This technique is commercially available and already used regularly, particularly for monitoring neonates.
However, it can be difficult to measure the local oxygenation of deeper structures, as it depends on light traveling through the tissue and returning to the sensor.
Oxygen electrodes can also be used, which can be inserted into the desired area for measurement.
These give a direct readout of the p\Otwo, in the tissue surrounding the probe.
However, inserting these oxygen electrodes into the tissue is extremely invasive, which has limited the use of these sensors.

Nuclear Magnetic Resonance (NMR) presents an alternative way to measure blood oxygen saturation.
It has been known since the early 1980s that the saturation of blood is linked to it's transverse relaxation time \Ttwo.
This relationship is caused by the change in the magnetic properties of haemoglobin when bound to oxygen.
This effect is a contributing factor to BOLD(Blood Oxygen Level Dependent) contrast which underlies fMRI(functional Magnetic Resonance Imaging).
Non-invasive measurement of local oxygen saturation has also been demonstrated using this effect.
These applications have been developed using high-field (>1.5 T) imaging systems, whose large size and cost means that these techniques are not in clinical use.

In the last 25 years, developments in portable magnetic resonance, using single sided magnets and coils, have meant that NMR can be used outside of laboratories and MRI suites.
These magnetic resonance instruments produce a sweet spot, where the magnetic field and coil sensitivity combine to produce an NMR signal from protons inside this region.
Combining this with the oxygen dependent \Ttwo change could produce an instrument which is sensitive to local changes in tissue oxygenation.
However, single sided NMR systems have limitations on the strength and homogeneity of the magnetic field they produce, due to the use of permanent magnets, and the geometry of the system.
This means that it is important to understand how factors like field strength and homogeneity affect how oxygenation dependent \Ttwo changes can be observed in this type of system.

%Additionally, the magnitude of the change in oxygen saturation that can be detected using this magnetic resonance relaxation effect is also important, as being able to measure slight changes in oxygenation will mean that this technique can provide clinically relevant early warnings of low oxygenation.

In this thesis, this parameter space has been mapped out to determine how these factors affect the magnitude of the oxygenation \Ttwo effect in whole blood.
Chapter \ref{ch:background} presents background information on NMR relaxation, the tranport of oxygen in blood, and the origin of the oxygenation dependent \Ttwo change.
Chapter \ref{ch:exptsetup} describes the two versions of the experimental setup used for the two series of experiments in this thesis.
Chapter \ref{ch:stoppedflow} presents the results of stopped flow experiments into the \Ttwo change at a series of oxygenation steps, while \autoref{ch:cont} presents the results of experiments measuring \Ttwo changes continuously as \SOtwo is slowly decreased.
Chapter \ref{ch:models} evaluates an alternative model proposed by Jensen and Chandra to describe the process causing decreases in \Ttwo.
