\chapter{Background}\label{ch:background}

\section{NMR theory}
\begin{itemize}
\item Protons in a magnetic field precess at the Larmor frequency
\item T1 and \Ttwo relaxation
\item Bloch equation
\item Spin echo / CPMG refocusing
\item PGSE measurement of displacement by phase shift
\item NMR measurements in flow
\end{itemize}

\section{Blood and oxygen saturation measurement}
\begin{itemize}
\item oxygen transported in red blood cells
\item haemoglobin and deoxyhaemoglobin
\item pulse oximetry
\item oxygen microelectrode
\item \Ttwo changes due to oxygenation in blood.. LM model, then present JC as alternative theory in \autoref{ch:models}?

\end{itemize}

\subsection{\Ttwo changes due to oxygenation in blood}
\label{sec:back-TtwoSOtwo}
As mentioned above, changes in oxygenation cause the fraction of haemoglobin bound to Oxygen to change.
Decreasing the oxygenation means there is more deoxy-haemoglobin which, due to its increased paramagnetism, causes a larger susceptibility change between the red blood cell cytoplasm and the surrounding plasma.
While the change between intracellular and extracellular is difficult to measure, this change can be measured for whole samples of blood, and found to be $\Delta\chi_{DO} = 0.27 \mathrm{ppm  (cgs)}$ \cite{JainInvestigatingmagneticsusceptibility2012}.
Changes in susceptibility cause variations in the magnetic field, which causes the refocusing pulses in the CPMG experiment to be less effective in recovering the phase.
This causes increased dephasing, which is observed as a decreased \Ttwo.

In the literature, the size of this decrease is typically described using the Luz-Meiboom equation.
This equation comes from the study of a chemical exchange process, where protons on an ammonium ion exchange with the solvent\cite{LuzNuclearMagneticResonance1963}.
Protons bound to the ammonium ion have a different chemical shift, which combined with the exchange, causes increased dephasing and a shorter \Ttwo.
Luz and Meiboom show that this process leads to a \Ttwo decrease dependent on the echo time given by \autoref{eq:LMchemEx}\cite{LuzNuclearMagneticResonance1963}, where $p_i$ is the fraction of protons in state $i$, $\delta_i$ is the shift of protons in state $i$, $t_{ec}$ is the echo time, and $\tau_{ex}$ is the average time between exchanges. It also includes the \TtwoO to include the \Ttwo when there is no exchange contribution.

Some authors have proposed that this is a similar situation to red blood cells, where the susceptibility change due to haemoglobin causes a difference in the field in the cytoplasm compared to the plasma, and protons exchange across the cell membrane\cite{BryantMagneticrelaxationblood1990}
Others have proposed that the dephasing is caused by diffusion through intracellular and/or extracellular gradients generated by the susceptibility change \cite{GomoriNMRRelaxationTimes1987,BrooksComparisont2relaxation1995,BrooksT2shorteningweaklymagnetized2001}.
The exchange time then becomes the time for spins to experience the range of fields in the gradient.

Wright\cite{WrightEstimatingoxygensaturation1991} applied this exchange model to blood, expressing it with more useful parameters. In \autoref{eq:LMblood} the summation over the states becomes $P_A$, which is the relative population of protons in cytoplasm and plasma (and therefore the haematocrit, and the frequency difference given by a term dependent on the $sO_2$, the field strength $\omega_0$, and a dimensionless factor $\alpha$ which is dependent on the susceptibility change of deoxyhaemoglobin.

\begin{equation}
\label{eq:LMchemEx}
\frac{1}{T_2} = \frac{1}{T_{20}} + \sum_i{p_i\delta_i^2} \left(1 - \frac{t_{ec}}{2\tau_{ex}} \tanh{ \frac{2\tau_{ex}}{t_{ec}} }\right)
\end{equation}

\begin{equation}
\label{eq:LMblood}
\frac{1}{T_2} = \frac{1}{T_{20}} + (P_A)(1 - P_A)\tau_{ex} \left[(1-sO_2)\alpha\omega_0\right]^2 \left(1 - \frac{2\tau_{ex}}{t_{ec}} \tanh{\frac{t_{ec}}{2\tau_{ex}} } \right)
\end{equation}

This exchange model is used in this research, as it is most commonly used in the literature.
An alternative model which more accurately characterise the diffusion of protons around red blood cells has also been developed\cite{JensenNMRrelaxationtissues2000}, and is investigated and compared with the exchange equation in \autoref{ch:models}.

%It is also possible that both mechanisms are occuring together, in which case, it is useful to know the relative contributions of the two processes.
%Based on studies of the spectroscopic line width of red blood cell suspensions, Matwiyoff et al. suggest that diffusion processes dominate at higher fields, while exchange between cytoplasm and plasma is more significant at lower fields (\textless 1.5 Tesla)\cite{Matwiyofflineshapeswater1990}.
