\chapter{Background}\label{ch:background}

\section{Nuclear Magnetic Resonance}

Nuclear Magnetic Resonance (NMR) is an effect most often associated with chemical spectroscopy, or magnetic resonance imaging.
This section will give a brief overview of the physics of NMR, using the semi-classical model and the Bloch equations.
It describes NMR occuring to protons (\textsuperscript{1}H), which is the most common nucleus used for NMR, and what is used in this thesis, although NMR can be observed with any nuclei with non-zero spin.
For a more complete discussion, the reader is directed to textbooks such as Abragham and Callaghan.

Protons have an magnetic moment due to their intrinsic angular momentum, or `spin'.
To observe nuclear magnetic resonance, protons are put into a magnetic field (\Bzero) that interacts with the protons magnetic moment, lowering the energy of magnetic moments aligned with \Bzero (along \textit{z}).
This makes it more energetically favourable for the protons' magnetic moment to align with the field, so that slightly more protons are aligned with the field.
This imbalance generates a net magnetisation aligned with the magnetic field, which is manipulated with the \textit{B\textsubscript{1}} field to generate the magnetic resonance signal.
As the size of the net magnetisation depends on the relative difference of spins towards and against the field, the net magnetisation becomes stronger when a larger \Bzero field is used.
Any imbalance created is typically very small, which means that the NMR signal generated is very weak.

A magnetic moment in a magnetic field experiences a torque, which causes it to rotate around an axis along the magnetic field.
This is called Larmor precession, and gives rise to the Larmor frequency $\omega$, which defines the rate of rotation.
Protons in a magnetic field precess at a frequency given by \autoref{eq:back-larmor}, where $\gamma$ is the gyromagnetic ratio (\SI{2.56e8}{rad/T} for protons).
\begin{equation}
\omega = \gamma B_0
\label{eq:back-larmor}
\end{equation}
At equilibrium, the net magnetisation is along \textit{z}, so it is not visible.

The effect of the \textit{B\textsubscript{1}} field can be seen by considering a frame (\textit{x', y', z'}) which rotates around the \textit{z} axis at this precession frequency (a `rotating frame', when compared to the stationary `lab frame')
In the rotating frame, the net magnetisation still lies along the \textit{z}-axis.
Applying an additional magnetic field, along \textit{x'} (i.e. perpendicular to \Bzero) to the \textit{z}-axis causes the magnetisation to rotate around \textit{x'} towards \textit{y'}, following the same precession principle.
The strength and time that the\Bone field is applied for can be changed to control the amount of rotation, leading to \SI{90}{\degree} and \SI{180}{\degree} pulses.

Transforming this into the lab frame means that the \Bone field becomes a magnetic field oscillating at the Larmor frequency,
while the components of the magnetisation along \textit{x} and \textit{y} correspond to a changing magnetic field generated by the sample.
This produces the resonance condition for NMR -- the frequency of the \Bone field needs to match the precession frequency to cause maximum rotation.
These can be created and detected using a coil, as the rotating magnetic field from the sample induces a voltage in the coil.
This is known as the Free Induction Decay.

\section{\Tone and \Ttwo relaxation}
The NMR signal from a sample decays due to relaxation with two main mechanisms.
It can be caused by the return of net magnetisation to equilibrium, or by a loss of coherence across a sample.

As mentioned above, the equilibrium state of a sample in a \Bzero field is for the net magnetisation to align with the \Bzero magnetic field.
When this is disturbed, for example, in an NMR experiment or when the sample is first placed in the \Bzero field, the protons in the sample exchange energy to return to the equilibrium.
This process, also known as spin-lattice relaxation, is described by \autoref{eq:back-Tone}, where M\textsubscript{0} is the equilibrium magnetisation, and the time constant for relaxation is \Tone.

\begin{equation}
\frac{\mathrm{d} M_z}{\mathrm{d} t} = \frac{1}{T_1} (M_0-M_z)
\label{eq:back-Tone}
\end{equation}

The \Tone of a sample can be determined by an inversion recovery experiment.
A \SI{180}{\degree} pulse is applied to the sample, and the magnetisation remaining is measured after a range of variable delay time ($t_d$ )by applying a \SI{90}{\degree} pulse and taking the intensity of the FID.
This can be used to fit a curve with the equation
\begin{displaymath}
M(t_d) = M_0 (1 - 2 e^{-frac{t_d}{T_1}})
\end{displaymath}

In most cases, the \Tone controls how rapidly an experiment can be repeated.
A general rule is to set the repetition time (\TR - the delay between experiments) to 5$\times$\Tone, so that the sample magnetisation recovers to \SI{>99}{\percent} of its original equilibrium level.
Otherwise the signal produced will be weaker, as this depends on magnetisation along the \textit{z} axis.
This also removes any remaining transverse magnetisation, as the protons return to equilibrium.
Because of this, \Ttwo < \Tone.

The \textit{x} and \textit{y} components of a sample's magnetisation (transverse magnetisation) also undergoes relaxation.
The transverse magnetisation is generated by the combination of magnetisation from protons across the sample.
The NMR signal is generated when the magnetic moments of all of the protons in a sample precess at the same rate, and are aligned in the same direction, creating coherence.
If protons in part of the sample experience a different magnetic field, they precess at a different rate and lose coherence with the rest of the sample.
This reduces the overall transverse magnetisation of the sample, which decreases following \autoref{eq:back-Ttwo}, where \Ttwo is the time constant for this relaxation.

\begin{equation}
\frac{\mathrm{d}M_{xy}}{\mathrm{d}t} = - \frac{1}{T_2} (M_{xy})
\label{eq:back-Ttwo}
\end{equation}

In a physical sample, protons are always experiencing different magnetic fields due to the fluctuating magnetic moments of surrounding protons.
This process of dephasing is random and non-reversible.
The physical properties of a sample control the amount of interaction between protons on surrounding molecules, and the rates of \Ttwo relaxation, for example, solids have extremely fast \Ttwo relaxation times, on the order of 10s of microseconds, while water can have a \Ttwo on the order of seconds.

\Ttwo relaxation is also be increased by the presence of paramagnetic ions in the sample, such as Cu\textsuperscript{2+}, Mn\textsuperscript{2+} and Gd\textsuperscript{3+}.
These affect the magnetic fields experienced by protons in the sample, and cause additional dephasing.
Their effect is characterised by a relaxivity \textit{r}, measured in \si{s^{-1}/mmol}.
This is due to the additive nature of relaxation rates: when multiple processes cause relaxation, the relaxation rates add together:

\begin{displaymath}
\frac{1}{\Ttwo} = R_2 = \frac{1}{T_{2 intrinsic}} + r_{2} c_{metal}
\end{displaymath}

In addition, \Ttwo relaxation is also associated with the homogeneity of the \Bzero field, as different magnetic fields across the sample will produce different precession rates, and also cause protons to dephase (although this can be recovered - see \autoref{sec:back-spinecho}).

The precession of the net magnetisation around the \Bzero field, and the effect of relaxation on the net magnetisation can be expressed using the phenomelogical Bloch equation \autoref{eq:back-Bloch}.

\begin{equation}
\frac{\mathrm{d}M}{\mathrm{d}t} = \gamma M \times B + \frac{1}{T_1} (M_0 - M_z) - \frac{1}{T_2} M_{xy}
\label{eq:back-Bloch}
\end{equation}

\section{Spin echoes and CPMG}
\label{sec:back-spinecho}
The spin echo was first described by Hahn in 1950. CITE hahn 1950
By applying a \SI{180}{\degree} pulse a time $\tau$ after a \SI{90}{\degree} pulse, he observed a signal forming after another $\tau$.
He showed that this is due to the spread of phase caused by the \Bzero field inhomogeneity being reversed, which allows the protons to rephase and regenerate a net transverse magnetisation that can be detected.

The formation of the spin echo is shown diagrammatically in TODO picture?.
The inhomogeneous \Bzero field causes protons across the sample to have different Larmor frequencies, meaning that they precess at different rates, and accumulate different amounts of phase as time continues.
The different phases cause the net transverse magnetisation of the whole sample to decay, with a shorter time constant \Ttwostar.
Applying a \SI{180}{\degree} refocussing pulse $\tau$ after the \SI{90}{\degree} pulse causes the magnetic moments of each proton to rotate through \SI{180}{\degrees} around the \textit{x} (or \textit{y}) axis, so that the different precession rates unwind the different accumulated phase.
Thus unwinding causes the magnitude of the net transverse magnetisation to increase, reaching a maximum at $2\tau$, which is the centre of the echo.
The spin echo will still have a reduced intensity, due to the intrinsic \Ttwo relaxation proceses described above.
However, it removes the effects of the inhomogeneous \Bzero field, and allows the measurement of the intrinsic \Ttwo relaxation process.

This rephasing process relies on the protons experiencing the same \Bzero field in the period before and after the \SI{180}{\degree} pulse.
If protons move across the inhomogeneous field during this time, they may not experience the same field and not be complete refocussed at $2\tau$.
This can be due to diffusion(the movement of protons in the sample) or processes such as chemical exchange (explored below).

Once the magnetisation has been refocused in the spin echo, the transverse magnetisation decays following the \Ttwostar relaxation time again.
Carr and Purcell proposed repeating the refocussing pulses, to create more echoes and further map out the \Ttwo relaxation. CITECarrPurcell
In this experiment, the refocussing pulses are evenly spaced by an echo time, as shown in TODO diagram.
Meiboom and Gil improved this experiment by adjusting the phase of the \SI{180}{\degree} pulses relative to the \SI{\90}{\degree} pulse, such that they rotate the magnetisation around the \textit{y}-axis rather than \textit{x}. CITEMeiboom
This reduces the errors caused by an incorrect \SI{180}{\degree} pulse.
With this modification, the CPMG experiment is commonly used fast and robust measurements of the \Ttwo of a sample.

\begin{equation}
S(t) = S_0 e^{\frac{t}{T_2}}
\label{eq:CPMGT2fitting}
\end{equation}
The intensity of echoes in a CPMG experiment follows a monoexponential decay (for samples with a single relaxation component).
The \Ttwo can be measured by the slope of a semi-log plot, or by non-linear fitting of the function \ref{eq:CPMGT2fitting}.
While measurements of the intrinsic \Ttwo should not be afffected by the echo time used, process like exchange and diffusion can alter the measured \Ttwo.
For example, for protons moving in a magnetic gradient, a longer echo time means that protons experience a wider range of \Bzero fields before being refocussed.
This makes the refocusing less efficient, and results in a shorter observed \Ttwo, following \autoref{eq:CPMGgrad}.
Similarly, exchange processes can cause protons to experience different magnetic fields in the timescales of the experiment, causing the refocussing to be less efficient.
These processes, and their applicability to blood are discussed further in \autoref{sec:back-T2SO2}.

\begin{equation}
\frac{1}{T_2} = \frac{1}{T_2} + \frac{\gamma^2 G^2 D \tau^2}{12}
\label{eq:CPMGgrad}
\end{equation}

\section{PGSE experiments}
\label{sec:back-PGSE}
Pulsed Gradient Spin Echo (PGSE) experiments combine the spin echo with magnetic field gradients to measure the displacement of protons in a sample.
Traditionally, a magnetic field gradient creates a known change in the \Bzero field, which is linearly related to its position: $B_0(z) = B_0 + g(z)$
This allows protons to be spatially resolved, as they will have different Larmor precession frequencies at different locations.
This principle is the basis of Magnetic Resonance Imaging (MRI).

Stejskal and Tanner proposed the PGSE experiment in 1965
Their experiment uses a pulsed magnetic field gradient, which creates a range of precession rates across the sample for a short time, so that protons accumulate a phase related to their position.
A \SI{180}{\degree} pulse is then applied, and an identical pulsed field gradient is reapplied.
This causes the phase accumulated in the first pulse to be reversed before the spin echo is formed.
Assuming there is no movement, the protons in the sample will be completely rephased, and the normal spin echo will be recovered.

Diffusion (random movement) of protons means that the rephasing is not as effective, as the phase accumulated in the first pulsed gradient will is not totally removed.
This means that some of the protons will cancel each other out, and cause the intensity of the spin echo to be reduced.
By running experiments with a range of gradient strengths (\textit{G}) and measuring the echo intensity, the diffusion coefficient (D) for the sample can be obtained using \autoref{eq:PGSEdiff}, where $\delta$ is the length of time the pulsed gradient field is on, $\Delta$ is the time between the two gradient pulses.

\begin{equation}
S(G) = S_0 e^{-\gamma^2 G^2 \delta^2 (\Delta - \frac{\delta}{3} ) D}
\labe{eq:PGSEdiff}
\end{equation}

Uniform motion of protons in a sample produces a different effect.
Rather than the intensity of the spin echo decreasing, the phase of the spin echo is shifted.
That is, the spin echo will form with the transverse magnetisation rotated slightly around the \textit{xy} plane, as the second gradient pulse `overcorrects' the dephasing.
The velocity of the sample can be determined using two methods. CITECallaghanXia
By measuring the phase and intensity at a range of gradient strengths, the velocity can by found using a Fourier transform.
Alternatively, taking the phase ($\phi$) of the echo measured with a single gradient strength, and comparing it to a reference where there is no gradient, the flow velocity can be measured using \autoref{eq:PGSEvel}.
This second method was used in these experiments, due to the increased speed.
\begin{equation}
v = \frac{\phi}{\gamma \Delta \delta g}
\label{eq:PGSEvel}
\end{equation}

\section{Blood and oxygen saturation}

Blood contains multiple components, including red blood cells, white blood cells, platelets, and plasma.
Red Blood Cells (RBCs) are associated with oxygen tranport, while white blood cells are part of the immune system.
Platelets are involved in the blood clotting process, and plasma is the water, nutrients and proteins which these components are suspended in.
Blood is pumped around the circulatory system by the heart, and transports nutrients to all parts of the body.
On average, blood makes up 7\% of a person's body weight, making a volume of about \SI{5}{L}.

In addition to the nutrients carried in the plasma,  blood is responsible for the tranport of oxygen from the lungs to the body.
Most of the oxygen transported is carried by red blood cells, with a small amount dissolved in blood plasma.
These are shown in figure TODO, and have a biconcave disc shape, with a mean diameter of \SI{7.8}{\micro\metre}, and mean thickness of \SI{2.5}{\micro\metre}.
RBCs contain no cell nucleus and are very deformable, which allows them to travel through narrow capillaries in tissue.
While they have no nucleus, they still metabolise slowly to maintain membrane deformability and ion transport through the cell membrane.
RBCs have a typical lifetime in the body of 120 days.

The percentage of blood volume in red blood cells is called the haematocrit.
It is typically around 0.4 in healthy men, and 0.36 in healthy women.
The primary role of RBCs is to transport the oxygen carrying protein haemoglobin in the blood, so they can contain very high concentrations of up to \SI{34}{g/L} of haemoglobin inside.

\section{Oxy- and deoxy-haemoglobin}
Haemoglobin is a protein which can reversibly bind to oxygen to improve transport in the body.
By binding the oxygen to haemoglobin, blood can transport 30-100 times more oxygen than relying on dissolved oxygen in the plasma.
It contains four heme groups, each containing an iron ion (Fe\textsuperscript{2+}) which can loosely bind to oxygen.
In the lungs, oxygen diffuses into the blood and RBCs, and is taken up by haemoglobin to form oxy-haemoglobin.
This increases the oxygen saturation (\SOtwo), which is defined as the fraction of all haemoglobin bound to oxygen in \autoref{eq:o2sat}.

\begin{equation}
sO_2 = \frac{f_{oxyHb}}{f_{oxyHb} + f_{deoxyHb} + f_{dysHb}}
\label{eq:o2sat}
\end{equation}

After leaving the lungs, the \SOtwo will be around 97\%, and the p\Otwo will be around \SI{95}{mmHg}.
As blood travels around the body, oxygen diffuses from the blood into the tissue, which has a lower p\Otwo due to metabolism.
The oxygen dissolved in plasma decreases, decreasing the p\Otwo (the partial pressure of oxygen).
This change triggers the oxy-haemoglobin to release oxygen, forming deoxy-haemoglobin and effectively buffering the p\Otwo change.
In venous blood, after leaving tissue, the \SOtwo will be between \SIrange{20}{70}{\percent}, and have a p\Otwo between \SIrange{20}{40}{mmHg} (depending on the demand for oxygen in the tissue).

The relationship between \SOtwo and p\Otwo is non-linear due to effects such as co-operative binding.
Binding the first oxygen molecule to the one of the heme groups causes a conformational change in the protein which makes binding at the other heme groups easier.
This makes the \SOtwo increase more steeply (as a function of p\Otwo) once the first oxygen is bound.
This effect also works in reverse, causing the protein to release oxygen.
Physiologcally, this causes a release of oxygen in areas with low p\Otwo, which helps to deliver oxygen to tissue that needs it.

Other factors also affect the relationship between \SOtwo and p\Otwo.
One example is the pH, where an increase in acidity (e.g. 7.4 to 7.2) causes an decrease in oxygen affinity, and the release of more oxygen from haemoglobin.
The Bohr effect relies on this dependency
pH in the blood is also affected by dissolved \COtwo, which exists in equilibrium with carbonic acid in the blood.
An increased p\COtwo causes more carbonic acid to be formed, lowering the pH of the blood.
This causes oxygen to be released in areas where \COtwo is being produced, due to metabolism.

Changes in the \SOtwo can be measured using the different optical and magnetic properties of oxy- and deoxy- haemoglobin.
The binding of oxygen to haemoglobin causes them to have slightly different optical absorption spectra.
In particular, the deoxy-haemoglobin absorbs more strongly in the red region of the spectrum.
This causes the colour change, where oxygenated blood appears red, and deoxygenated blood appears almost black.

Oxygen binding to haemoglobin also causes a change in the electronic structure of the iron atom.
In the oxygenated state, the iron atom complex has no unpaired electrons.
In the deoxygenated state however, the iron complex contains unpaired electrons, making it paramagnetic.
This increases the susceptiblity inside the RBC, which separates the haemoglobin from the plasma.
Changes in susceptibility across a sample produce inhomogeneity in the \Bzero magnetic field.
The inhomgeneous field generated by the susceptibility difference between RBC and plasma creates the decreasing \Ttwo effects used in this thesis.
The mechanism for this decrease is described in \autoref{sec:back-T2SO2} below.

There are also other MR methods for detecting changes in \SOtwo which rely on the oxy-/deoxy-haemogobin susceptibility change.
The Blood Oxygenation Level Dependent contrast used in functional imaging relies on the susceptibility change of blood in vessels in the brain.
The induced field inhomogeneity causes additional \Ttwostar relaxation, lowering the observed signal in each voxel.
By tracking the signal decrease, and increase due to inflow of oxygenated blood over time, regions of the brain using oxygen can be identified.
Quantitative Susceptibility Mapping techniques can also be used to directly measure the susceptibility change of blood vessels.
This gives a more direct measurement of \SOtwo, as and has recently been shown as a potential method for calibrating the \Ttwo effect described below.

\section{\SOtwo measurement}
The gold standard method for measuring the \SOtwo of blood is co-oximetry, which relies on the optical absorption changes in oxy-/deoxy- haemoglobin.
This method typically uses measurements of absorption at multiple wavelengths to identify the concentrations of oxy-haemoglobin, deoxy-haemoglobin and other dys-haemoglobins in a blood sample.

Alternatively, the \SOtwo can be found using knowledge of the p\Otwo of the sample, and the relationship between p\Otwo and \SOtwo.
The p\Otwo of a sample can be measured using a Clark electrode, which uses a redox reaction with a rate dependent on the p\OTwo.
The Clark electrode contains a platinum/silver cathode/anode pair behind an oxygen permeable membrane.
Oxygen is reduced to water following the chemical equation below, and the current generated in the electrode is therefore proportional to p\OTwo.
This method is used in the iStat in this thesis to measure \SOtwo.

\begin{displaymath}
\mathrm{O_2} + \mathrm{4e}^{-} + \math{4 H^{+}} \rightarrow \mathrm{2 H_{2}O}
\end{displaymath}

These methods both require taking samples of blood to measure.
Pulse oximetry is a non-invasive technique which is commonly used to measure \SOtwo in patients.
A pulse oximeter shines light through a part of the patient, typically a finger, and measures the light transmitted or reflected from the tissue.
The absorbance from the tissue can be rejected with the knowledge that the arterial blood flow is pulsatile, meaning that the component of absorption varying in time can be assigned to the blood.
The differing absorbance at multiple wavelengths is used to find the fraction of oxy-haemoglobin in blood.

While in theory, this could be done using the Beer-Lambert law with known extinction coefficients of oxy- and deoxy-haemoglobin, the effect of scattering  from red blood cells mean that more empirical calibration methods are used to convert the measured light intensities, and the \SOtwo.
In two wavelength pulse oximeters, like used in this study, a quadratic calibration curve is used to approximate the relationship between the ratio of light intensities \textit{R}, and the \SOtwo.
Pulse oximeters with more wavelengths can apply more advanced algorithms to measure \SOtwo.

\section{\Ttwo changes due to oxygenation in blood}
\label{sec:back-TtwoSOtwo}
As mentioned above, changes in oxygenation cause the fraction of haemoglobin bound to oxygen to change.
Decreasing the oxygenation means there is more deoxy-haemoglobin which, due to its increased paramagnetism, causes a larger susceptibility change between the red blood cell cytoplasm and the surrounding plasma.
While the change between intracellular and extracellular is difficult to measure, this change can be measured for whole samples of blood, and found to be $\Delta\chi_{DO} = 0.27 \mathrm{ppm  (cgs)}$ \cite{JainInvestigatingmagneticsusceptibility2012}.
Changes in susceptibility cause variations in the magnetic field, which causes the refocusing pulses in the CPMG experiment to be less effective in recovering the phase.
This causes increased dephasing, which is observed as a decreased \Ttwo.

In the literature, the size of this decrease is typically described using the Luz-Meiboom equation.
This equation comes from the study of a chemical exchange process, where protons on an ammonium ion exchange with the solvent\cite{LuzNuclearMagneticResonance1963}.
Protons bound to the ammonium ion have a different chemical shift, which combined with the exchange, causes increased dephasing and a shorter \Ttwo.
Luz and Meiboom show that this process leads to a \Ttwo decrease dependent on the echo time given by \autoref{eq:LMchemEx}\cite{LuzNuclearMagneticResonance1963}, where $p_i$ is the fraction of protons in state $i$, $\delta_i$ is the shift of protons in state $i$, $t_{ec}$ is the echo time, and $\tau_{ex}$ is the average time between exchanges. It also includes \TtwoO to include the \Ttwo when there is no exchange contribution.

\begin{equation}
\label{eq:LMchemEx}
\frac{1}{T_2} = \frac{1}{T_{20}} + \sum_i{p_i\delta_i^2} \left(1 - \frac{t_{ec}}{2\tau_{ex}} \tanh{ \frac{2\tau_{ex}}{t_{ec}} }\right)
\end{equation}

Some authors have proposed that this is a similar situation to red blood cells, where the susceptibility change due to haemoglobin causes a difference in the field in the cytoplasm compared to the plasma, and protons exchange across the cell membrane\cite{BryantMagneticrelaxationblood1990}
Others have proposed that the dephasing is caused by diffusion through intracellular and/or extracellular gradients generated by the susceptibility change \cite{GomoriNMRRelaxationTimes1987,BrooksComparisont2relaxation1995,BrooksT2shorteningweaklymagnetized2001}.
The exchange time then becomes the time for spins to experience the range of fields in the gradient.

Wright\cite{WrightEstimatingoxygensaturation1991} applied this exchange model to blood, expressing it with more useful parameters.
In \autoref{eq:LMblood} the summation over the states becomes $P_A$, which is the relative population of protons in cytoplasm and plasma (and therefore the haematocrit, and the frequency difference given by a term dependent on the $sO_2$, the field strength $\omega_0$, and a dimensionless factor $\alpha$ which is dependent on the susceptibility change of deoxy-haemoglobin, and the geometry of the red blood cell.

\begin{equation}
\label{eq:LMblood}
\frac{1}{T_2} = \frac{1}{T_{20}} + (P_A)(1 - P_A)\tau_{ex} \left[(1-sO_2)\alpha\omega_0\right]^2 \left(1 - \frac{2\tau_{ex}}{t_{ec}} \tanh{\frac{t_{ec}}{2\tau_{ex}} } \right)
\end{equation}

Jensen and Chandra investigated this problem using the weak-field approximation to derive expressions for the signal in a CPMG experiment from protons diffusing in a weakly inhomogeneous field\cite{JensenNMRrelaxationtissues2000}.
This method uses a correlation function $K(t)$ that describes the variations in the magnetic field as protons diffuse.
The true correlation function is normally not known analytically however, so approximations are used.
By approximating this correlation function with a simple exponential decay \autoref{eq:JCExpCorr}, they found that the relaxation rate in a CPMG experiment is given by \autoref{eq:LMsimp} \cite{JensenNMRrelaxationtissues2000}.

\begin{equation}
K(t) = K_0 e^{-t/\tau}
\label{eq:JCExpCorr}
\end{equation}

\begin{equation}
\label{eq:LMsimp}
\frac{1}{T_2} = \frac{1}{T_{20}} + \gamma^2 K_0 \tau_{ex} (1 - \frac{2\tau_{ex}}{t_{ec}} \tanh{\frac{t_{ec}}{2\tau_{ex}}})
\end{equation}

This has the same form as the Luz-Meiboom equation, but with a more general ``correlation time'', and the factor $K_0$ representing the variance of the magnetic field.
Because of this, the correlation time is not directly connected to an exchange process, which could explain why the Luz-Meiboom formula agrees with experimental results, but also gives a range of exchange times.
In practice, this \Kzero parameter describes how dependent the \Ttwo is on echo time.
A larger value of \Kzero means that the inhomogeneities are stronger, and cause a larger dephasing effect as the echo time increases.

This exchange model is used in this research, as it is most commonly used in the literature.
An alternative model which more accurately characterises the diffusion of protons around red blood cells has also been developed\cite{JensenNMRrelaxationtissues2000}, and is investigated and compared with the exchange equation in \autoref{ch:models}.
