\chapter{Conclusions and Future Work}\label{ch:conc}

These results confirm that changes in \Ttwo due to blood oxygenation can be detected at low fields in this \textit{in-vitro} system.
The continuous flow experiments showed that this \Ttwo effect follows the trends predicted by the Luz-Meiboom equation, with the increase in \Rtwo and \Kzero being proportional to (1-\SOtwo)\textsuperscript{2} and to \Bzero\textsuperscript{2}.
While some experimental parameters will need to be better controlled in future experiments, the effects of temperature and flow rate do not cause signficant uncertainty in the results.
A decrease in \Ttwo also occured over the course of each experiment, and experiments using UV/Vis spectroscopy suggested this was due to haemolysis occuring in the blood.
By applying the Luz-Meiboom model, measurements using different echo times were used to remove the effect of the background \Ttwo change in the blood samples, allowing for good tracking of the blood \SOtwo.

In a separate series of experiments, the effect of echo time on the \Ttwo decrease was tested.
The Luz-Meiboom equation worked well to describe this effect, although the Jensen and Chandra formula produced better agreement at all field strengths.
This agrees with findings in the literature.

This series of experiments will need to be repeated to become more robust, as there is only a single blood sample used in each experiment for the 5 field strengths.
Comparing the dependence of \Ttwo, \Rtwo and \Kzero on \SOtwo across these samples and field strengths will provide more confidence in the accuracy of these experimental results.
With more experiments, it will also be possible to investigate the effect of haematocrit, which has not been tested in this research.
Experiments with concentrated or diluted blood samples will allow the effect of haematocrit to be isolated from the effect of field strength.

While the results show that changes in blood oxygenation can be detected, there may still be more challenges before this technique can be applied \textit{in-vivo}.
For example these experiments detected the \Ttwo change in blood, but an \textit{in-vivo} system will need to detect the \Ttwo change in blood while in tissue.
Creating a magnet system will also present difficulties, as it can be difficult to optimise parameters such as field strength, field homogeneity and the sweet spot size and depth in a portable NMR system.
It will be important to know what compromises can be made in the design of a real system, while still maintaining good sensitivity to the \Ttwo contrast.

Other types of MRI contrast and techniques also have the potential to be applied in low field portable NMR systems.
In high-field systems, techniques such as diffusion-weighted imaging can be used to detect the early signs of damage to the brain.
Recently, the gradients produced by the NMR MOLE were used to provide accurate measurements of the diffusion coefficients in samples of liquid.
The detection of perfusion is also a clinically valuable measurement that could be implemented with a low field system.

These results show that the \Ttwo changes in blood due to changes in blood oxygenation are visible at low field, and serve as a foundation for the future development of a new low field device.
