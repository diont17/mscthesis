\documentclass[12pt, a4paper, twoside, openright]{book}

\usepackage{vuwthesis} % sets up some local things, mostly the front page

%\usepackage{palatino} % sets palatino as the default font
\usepackage{fontspec}
\setmainfont{TeX Gyre Pagella}
\usepackage{url} % for typesetting urls

\usepackage[colorlinks=true, linkcolor=black, urlcolor=black,citecolor=black]{hyperref}

\usepackage{xspace}
\usepackage{graphicx}
\usepackage{caption}
\usepackage{subcaption}
\usepackage{siunitx}
\usepackage[table]{xcolor}
\usepackage[bottom=3.0cm, left=3cm, right=3.0cm]{geometry}

\usepackage{pdflscape}
\usepackage{acronym}
%\renewcommand{\baselinestretch}{1.00}


\newcommand{\Ttwo} {\textit{T\textsubscript{2}}\xspace}
\newcommand{\SOtwo} {\textit{sO\textsubscript{2}}\xspace}
\newcommand{\TtwoO} {\textit{T\textsubscript{20}}\xspace}
\newcommand{\Kzero} {\textit{K\textsubscript{0}}\xspace}
\newcommand{\Gzero} {\textit{G\textsubscript{0}}\xspace}
\newcommand{\Otwo} {O\textsubscript{2}\xspace}
\newcommand{\Ntwo} {N\textsubscript{2}\xspace}
\newcommand{\COtwo} {CO\textsubscript{2}\xspace}
\newcommand{\TR} {\textit{T\textsubscript{R}}\xspace}
\newcommand{\Texc} {\textit{τ\textsubscript{ex}}\xspace}
\newcommand{\rc} {\textit{r\textsubscript{c}}\xspace}
\newcommand{\Tech} {\textit{t\textsubscript{ec}}\xspace}
\newcommand{\Rtwo} {\textit{R\textsubscript{2}}\xspace}
\newcommand{\Bzero} {\textit{B\textsubscript{0}}\xspace}
\newcommand{\Bone}  {\textit{B\textsubscript{1}}\xspace}
\newcommand{\Ttwostar} {\textit{T\textsubscript{2}\**}\xspace}
\newcommand{\Tone} {\textit{T\textsubscript{1}}\xspace}
\newcommand{\Hct} {\textit{Hct}\xspace}

\sisetup{separate-uncertainty = true}
%\sisetup{unit-color = blue}
%\sisetup{number-color = blue}

\begin{document}

\frontmatter
% Book style knows about front matter
% Report style doesn't so you need to set roman numbering etc yourself :-(

%%%%%%%%%%%%%%%%%%%%%%%%%%%%%%%%%%%%%%%%%%%%%%%%%%%%%%%
\title{Parameter Space Mapping for Blood Oxygenation Measurement with Low Field NMR}
\author{Dion Gary Thomas}

\subject{Physics}
\abstract{
Blood oxygenation is a critical physiological parameter for patient health.
The clinical importance of this parameter means that measurement of blood oxygenation is a routine part of care.
Magnetic resonance provides a way to measure blood oxygenation through the paramagnetic effect of deoxy-haemoglobin, which decreases the \Ttwo relaxation time of blood.
This effect has been well characterised at high fields (\SI{>1.5}{T}) for use in Magnetic Resonance Imaging, and it is a contributing factor to the Blood Oxygenation Level Dependent contrast used in functional MRI.
However there are relatively few studies of this effect at low magnetic fields, and these have only looked at extreme levels of oxygenation/deoxygenation.
To study this effect for potential application in a low-field device, we measured this effect to determine how factors such as oxygenation, field strength and CPMG echo time affect the \Ttwo of blood.

A continuous flow circuit, similar to a cardiopulmonary bypass circuit, was used to control parameters such as oxygen saturation and temperature, before the blood sample flowed into a variable field magnet (set at fields between 5-40 MHz), where a series of CPMG experiments with echo times ranging from \SIrange{1}{20}{ms} were performed to measure the \Ttwo.
Additionally, the oxygen saturation was continually monitored by an optical sensor, for comparison with the \Ttwo changes.
This allowed us to test the sensitivity of this effect at low fields.

These results show that at low fields, the \Ttwo relaxation change still follows the trends shown in the literature, with a dependence on \Bzero squared, and on the fraction of deoxyhaemoglobin squared.
Additionally, these results were also compared with two theoretical models for the dependence on echo time, which have previously been tested at high fields: the Luz-Meiboom equation, and the Jensen and Chandra model.
Both models gave good agreement with the data measured at low fields.
These experiments show that the \Ttwo changes in blood due to oxygenation are still visible at low field, and that this technique should be feasible in a low field device.


%Changes in blood oxygen saturation change the magnetic properties of red blood cells, generating an inhomogeneous magnetic field which decreases the \Ttwo relaxation time of the water protons in blood.
%This effect has been well characterised at imaging fields (\SI{>1.5}{T}), and has been applied to measure changes in blood oxygen saturation \textit{in-vivo}, but at the low fields required for portable NMR devices, there are only a handful of studies in the literature.
%These showed that the size of the \Ttwo change increases with the field strength squared, althuoguh these experiments at low fields only looked at extreme levels of oxygenation/deoxygenation.

%At these field strengths and echo times, both models provided good agreement with the experimental data, however the Jensen and Chandra model model was able to provide better fits to the data.
}
% Books don't normally have abstracts, and this is a bit of a hack

% Uncomment the appropriate degree
%\phd
\mscthesisonly
%\mscwithhonours
%\mscbothparts
% \otherdegree{DEGREE OR DIPLOMA NAME}



%%%%%%%%%%%%%%%%%%%%%%%%%%%%%%%%%%%%%%%%%%%%%%%%%%%%%%%
\hypersetup{pdftitle=\@title}



\maketitle

\chapter*{Acknowledgments}\label{ch:ack}
Firstly I need to thank my supervisors: Sergei Obruchkov, Shieak Tzeng and Petrik Galvosas for their excellent supervision, and for giving me the opportunity to pursue this research.
In particular I would like to thank them for their encouragement when the baby-MRI decided to choke a month before I was due to give a talk about results collected on it.
I would like to think that the panic and stress that caused was worth it.

Another big thanks goes to the other members and friends of the VUW NMR group over the last 2 years.
Thank you for making this such a fun place to work, and for all of your help and friendship.
To my other past supervisors, Marcel and Tim, thank you for introducing me to NMR and for letting me spend my honours year here.

Next, I would like to thank my fantastic friends and flatmates.
Thanks to them for putting up with my work stories, and for helping to keep me sane.
Of course, I must also thank Wilbur, who is somehow always excited to see me get home.

I would not have been able to complete this work without the care and support I have received from my family.
Thanks to my Mum and Dad for their encouragement, advice and support, and for treating me to weekend brunches.
And to my brother Adrian for his motivated assistance in acquiring research materials.
Thanks also to my family in the Philippines, especially my Lola, for their support and prayers.

I gratefully acknowledge funding support from the New Zealand Ministry of Business, Innovation and Employment, and the Australian and New Zealand Society for Magnetic Resonance for a travel grant to present this work at the ANZMAG conference.

I look forward to more experimenting in the future

\addtocontents{toc}{\vskip-1.0cm}

\tableofcontents

\listoffigures
\listoftables
\chapter*{List of Acronyms}
\begin{acronym}
  \acro{BOLD}{Blood Oxygenation Level Dependent}
  \acro{CPB}{Cardiopulmonary Bypass}
  \acro{CPMG}{Carr Purcell Meiboom Gil}
  \acro{FID}{Free Induction Decay}
  \acro{fMRI}{functional MRI}
  \acro{Hct}{Haematocrit}
  \acro{MOLE}{Mobile Lateral Explorer}
  \acro{MR}{Magnetic Resonance}
  \acro{MRI}{Magnetic Resonance Imaging}
  \acro{NIRS}{Near-Infrared Spectrocsopy}
  \acro{NMR}{Nuclear Magnetic Resonance}
  \acro{PGSE}{Pulsed Gradient Spin Echo}
  \acro{RBC}{Red Blood Cell}

\end{acronym}

%%%%%%%%%%%%%%%%%%%%%%%%%%%%%%%%%%%%%%%%%%%%%%%%%%%%%%%

% book style knows about mainmatter
% if you are using report style you will have to rest page numbering etc.
\mainmatter

%%%%%%%%%%%%%%%%%%%%%%%%%%%%%%%%%%%%%%%%%%%%%%%%%%%%%%%

% individual chapters included here

\chapter{Introduction}\label{ch:intro}

Blood oxygenation is a critical parameter for assessing patient health and the management of disease.
Oxygen saturation measurement are a routine part of care, and changes in these measurements can act as warning signs of acute/serious problems in the body.
Low levels of Oxygen can cause tissue damage and cell death, particularly in sensitive tissue such as the brain, where even brief periods of
hypoxia can have catastrophic effects.
Low Oxygen levels can also be an issue for neonates, who can have circulatory and pulmonary systems which are still developing.
Being able to detect changes in oxygenation quickly means that action can be taken to limit and/or prevent ill effects to patients.

The typical method for measuring blood oxygenation in patients is through pulse oximetry, a non-invasive method which measures the change in absorbance at various wavelengths of light to detect the presence of oxygenated and deoxygenated haemoglobin.
This method can produce quick measurements of the blood oxygen saturation, while also being relatively cheap and robust.
As it measures the blood, this technique provides systemic information about Oxygen saturation, measured at a single point on the finger or ear lobe.
However, these saturation values are not necessarily representative of Oxygen levels in specific parts of the body.
Problems with blood flow (ischemia) to sensitive areas, such as the brain, can cause significant damage, and will not necessarily show up on pulse oximetry.
Because of this, local tissue oxygenation measurements are desired by physicians, but they are difficult to measure.

Techniques for measuring local tissue oxygenation in the brain include the use of NIRS (Near-Infrared Spectroscopy), and the use of oxygen measuring
polarographic electrodes.
NIRS is a non-invasive method where the changes in light absorption from oxy/deoxy-haemoglobin are observed at near-infrared wavelengths, which can penetrate
into tissue, similar to pulse oximetry.
This technique is commercially available and already used regularly, particularly for monitoring neonates.
However, it can be difficult to measure the local oxygenation of deeper structures, as it depends on light traveling through the tissue and returning to the sensor.
Oxygen electrodes can also be used, which can be inserted into the desired area for measurement.
These give a direct readout of the p\Otwo, in the tissue surrounding the probe.
However, inserting these oxygen electrodes into the tissue is extremely invasive, which has limited the use of these sensors.

Nuclear Magnetic Resonance (NMR) presents an alternative way to measure blood oxygen saturation.
It has been known since the early 1980s that the saturation of blood is linked to it's transverse relaxation time \Ttwo.
This relationship is caused by the change in the magnetic properties of haemoglobin when bound to oxygen.
This effect is a contributing factor to BOLD(Blood Oxygen Level Dependent) contrast which underlies fMRI(functional Magnetic Resonance Imaging).
Non-invasive measurement of local oxygen saturation has also been demonstrated using this effect.
These applications have been developed using high-field (>1.5 T) imaging systems, whose large size and cost means that these techniques are not in clinical use.

In the last 25 years, developments in portable magnetic resonance, using single sided magnets and coils, have meant that NMR can be used outside of laboratories and MRI suites.
These magnetic resonance instruments produce a sweet spot, where the magnetic field and coil sensitivity combine to produce an NMR signal from protons inside this region.
Combining this with the oxygen dependent \Ttwo change could produce an instrument which is sensitive to local changes in tissue oxygenation.
However, single sided NMR systems have limitations on the strength and homogeneity of the magnetic field they produce, due to the use of permanent magnets, and the geometry of the system.
This means that it is important to understand how factors like field strength and homogeneity affect how oxygenation dependent \Ttwo changes can be observed in this type of system.

%Additionally, the magnitude of the change in oxygen saturation that can be detected using this magnetic resonance relaxation effect is also important, as being able to measure slight changes in oxygenation will mean that this technique can provide clinically relevant early warnings of low oxygenation.

In this thesis, this parameter space has been mapped out to determine how these factors affect the magnitude of the oxygenation \Ttwo effect in whole blood.
\autoref{ch:background} presents background information on NMR relaxation, the tranport of oxygen in blood, and the origin of the oxygenation dependent \Ttwo change.
\autoref{ch:exptsetup} describes the two versions of the experimental setup used for the two series of experiments in this thesis.
\autoref{ch:stoppedflow} presents the results of stopped flow experiments into the \Ttwo change at a series of oxygenation steps, while \autoref{ch:cont} presents the results of experiments measuring \Ttwo changes continuously as \SOtwo is slowly decreased.
\autoref{ch:models} evaluates an alternative model proposed by Jensen and Chandra to describe the process causing decreases in \Ttwo\cite{JensenNMRrelaxationtissues2000}.

\chapter{Background}\label{ch:background}

\section{NMR theory}
\begin{itemize}
\item Protons in a magnetic field precess at the Larmor frequency
\item T1 and \Ttwo relaxation
\item Bloch equation
\item Spin echo / CPMG refocusing
\item PGSE measurement of displacement by phase shift
\item NMR measurements in flow
\end{itemize}

\section{Blood and oxygen saturation measurement}
\begin{itemize}
\item oxygen transported in red blood cells
\item haemoglobin and deoxyhaemoglobin
\item pulse oximetry
\item oxygen microelectrode
\item \Ttwo changes due to oxygenation in blood.. LM model, then present JC as alternative theory in \autoref{ch:models}?

\end{itemize}

\subsection{\Ttwo changes due to oxygenation in blood}
\label{sec:back-TtwoSOtwo}
As mentioned above, changes in oxygenation cause the fraction of haemoglobin bound to Oxygen to change.
Decreasing the oxygenation means there is more deoxy-haemoglobin which, due to its increased paramagnetism, causes a larger susceptibility change between the red blood cell cytoplasm and the surrounding plasma.
While the change between intracellular and extracellular is difficult to measure, this change can be measured for whole samples of blood, and found to be $\Delta\chi_{DO} = 0.27 \mathrm{ppm  (cgs)}$ \cite{JainInvestigatingmagneticsusceptibility2012}.
Changes in susceptibility cause variations in the magnetic field, which causes the refocusing pulses in the CPMG experiment to be less effective in recovering the phase.
This causes increased dephasing, which is observed as a decreased \Ttwo.

In the literature, the size of this decrease is typically described using the Luz-Meiboom equation.
This equation comes from the study of a chemical exchange process, where protons on an ammonium ion exchange with the solvent\cite{LuzNuclearMagneticResonance1963}.
Protons bound to the ammonium ion have a different chemical shift, which combined with the exchange, causes increased dephasing and a shorter \Ttwo.
Luz and Meiboom show that this process leads to a \Ttwo decrease dependent on the echo time given by \autoref{eq:LMchemEx}\cite{LuzNuclearMagneticResonance1963}, where $p_i$ is the fraction of protons in state $i$, $\delta_i$ is the shift of protons in state $i$, $t_{ec}$ is the echo time, and $\tau_{ex}$ is the average time between exchanges. It also includes the \TtwoO to include the \Ttwo when there is no exchange contribution.

Some authors have proposed that this is a similar situation to red blood cells, where the susceptibility change due to haemoglobin causes a difference in the field in the cytoplasm compared to the plasma, and protons exchange across the cell membrane\cite{BryantMagneticrelaxationblood1990}
Others have proposed that the dephasing is caused by diffusion through intracellular and/or extracellular gradients generated by the susceptibility change \cite{GomoriNMRRelaxationTimes1987,BrooksComparisont2relaxation1995,BrooksT2shorteningweaklymagnetized2001}.
The exchange time then becomes the time for spins to experience the range of fields in the gradient.

Wright\cite{WrightEstimatingoxygensaturation1991} applied this exchange model to blood, expressing it with more useful parameters. In \autoref{eq:LMblood} the summation over the states becomes $P_A$, which is the relative population of protons in cytoplasm and plasma (and therefore the haematocrit, and the frequency difference given by a term dependent on the $sO_2$, the field strength $\omega_0$, and a dimensionless factor $\alpha$ which is dependent on the susceptibility change of deoxyhaemoglobin.

\begin{equation}
\label{eq:LMchemEx}
\frac{1}{T_2} = \frac{1}{T_{20}} + \sum_i{p_i\delta_i^2} \left(1 - \frac{t_{ec}}{2\tau_{ex}} \tanh{ \frac{2\tau_{ex}}{t_{ec}} }\right)
\end{equation}

\begin{equation}
\label{eq:LMblood}
\frac{1}{T_2} = \frac{1}{T_{20}} + (P_A)(1 - P_A)\tau_{ex} \left[(1-sO_2)\alpha\omega_0\right]^2 \left(1 - \frac{2\tau_{ex}}{t_{ec}} \tanh{\frac{t_{ec}}{2\tau_{ex}} } \right)
\end{equation}

This exchange model is used in this research, as it is most commonly used in the literature.
An alternative model which more accurately characterise the diffusion of protons around red blood cells has also been developed\cite{JensenNMRrelaxationtissues2000}, and is investigated and compared with the exchange equation in \autoref{ch:models}.

%It is also possible that both mechanisms are occuring together, in which case, it is useful to know the relative contributions of the two processes.
%Based on studies of the spectroscopic line width of red blood cell suspensions, Matwiyoff et al. suggest that diffusion processes dominate at higher fields, while exchange between cytoplasm and plasma is more significant at lower fields (\textless 1.5 Tesla)\cite{Matwiyofflineshapeswater1990}.

\chapter{Experimental Setups}
\label{ch:exptsetup}

In all experiments in this thesis, a flow circuit was used to contain the blood and provide control over parameters like oxygenation.
This circuit is similar to what is used in cardio-pulmonary bypass surgery, where a patient's blood is pumped and oxygenated by a heart lung machine.
Over the course of this research, the experimental setup was improved, leading to two main versions - the original stopped flow setup, and the continuous flow setup.

\section{Stopped flow experimental setup}
\label{sec:exptsetup-stopflow}

\subsection{Flow circuit and pump}
\begin{figure}
\centering
\includegraphics[width=0.8\textwidth]{figures/exptsetup/BloodMixingSetup.pdf}
\caption{Schematic of the Stopped flow setup}
\label{fig:exptsetup-stopflowschematic}
\end{figure}

A schematic of the flow stopped setup is shown in \autoref{fig:exptsetup-stopflowschematic}.
The main components are the roller pump and the oxygenation membrane, with the blood collection bag acting as a reservoir.
These are all joined by 1/4'' medical grade PVC tubing.
The total volume of this circuit was typically \SI{60}{\milli\litre}, with the blood bag containing 450 ml TODO check logbook.

The roller pump is used to generate flow around the circuit, by rotating the pump head to drive 2 rollers that squeeze the walls of the tube and force the blood along.
The pump is also designed for use in CPB surgery, and is a Stockert SIII, with a flow rate controllable from \SIrange{0.01}{6}{L/m} when used with the 1/4'' tube. TODOcheck?.
These versions of the flow circuit included a bypass section controlled by clamping the tubes so that the blood flow could be directed to around the reservoir, allowing for faster changes in the oxygenation due to a lower effective volume.
One important setting on the pump is the occlusion, which controls how much the tubing is squeezed as the pump head rotates.
Having the occlusion set too loosely causes the pump to become ineffective, as the blood can flow backwards through the pump, while setting it too tightly was found to cause damage to the red blood cells as they travel through the pump.
One problem with the model of pump used in these experiments does not have any sort of readout of the occlusion setting, so this had to be set at the correct value by following a testing procedure each time the occlusion was changed.

\subsection{Oxygenator and Gas Mixer}
A hollow-fibre membrane oxygenator was used to control the oxygenation of the blood in the circuit.
This was a Medtronic Affinity Pixie oxygenator, designed for use in paediatric CPB procedures.
It allows for gas exchange and temperature control of the blood flowing through, with the oxygenation controlled by the mix of gases going into the oxygenator.
This paediatric model was chosen to minimize the required priming volume, and it still provided more than enough capacity to oxygenate the blood at the flow rates used.
The temperature of the blood was regulated by flowing water from a temperature controlled waterbath through the oxygenator, which has a separate path built into it for this purpose.
The water bath temperature was set to \SI{35}{\degreeCelsius}, although this corresponded with a blood temperature at the outlet of \SI{33}{\degreeCelsius} at maximum flow rates (with lower temperatures at lower flow rates.)

The gas mix was set using a Dansensor MapMix 3 gas mixer, which allowed control of the different proportions of \Ntwo, \Otwo, and \COtwo flowing through the oxygenator.
The Oxygen fraction was typically varied from 0\% up to 21\%, the value of atmospheric air.
Nitrogen was used as a non-Oxygen containing gas, and between 2.5-5\% \COtwo was also added to ensure the pH of the blood sample remained stable over the course of the experiment (as these are linked by the bicarbonate buffer system in blood).
Gases for all experiments were used as supplied from BOC and were instrument grade or higher.
The output of the gas mixer flowed into a pressurised buffer tank before flowing into a Dansensor MapCheck Provectus gas analyser, which allowed us to check the fraction of the three gases which flowed into the oxygenation membrane.
This gas analyser also provided some flow regulation, although this was supplemented with an adjustable flow valve.
Typically, the gas flow rate was set to only \SIrange[per-mode=symbol]{1}{2}{\liter\per\minute} to avoid creating bubbles in the oxygenator.

\subsection{Blood collection}
Samples of whole blood were collected by venipuncture from healthy volunteers (taking place at Otago medical school).
This procedure was approved by the central health and disability ethics committee, and all volunteers provided written, informed consent.
Approximately \SI{450}{ml} of blood was collected into a blood bag containing \SI{66.5}{\milli\litre} CPD anticoagulant/preservative solution (Haemonetics Leukotrap WB system).
CPD solution is rated for storage of red blood cells for up to at least 3 weeks of storage\cite{Hessupdatesolutionsred2006}.
Blood samples were held at room temperature following collection, before undergoing leukoreduction filtering to remove white blood cells.
As it has been shown that the presence of white blood cells can adversely affect the condition of red blood cells in storage\cite{Hessupdatesolutionsred2006}.
Samples were then moved to a refrigerator and stored at \SI{4}{\degreeCelsius} until required for experiments (up to 30 TODO check! days for stopped flow experiments, up to 10 days for continuous flow experiments)

\subsection{NMR setup}
For these experiments, 3 different permanent magnet systems were used to obtain data at 3 different field strengths.
TODO pics of magnets!!

The first system was a 12 MHz (\SI{0.3}{T}) Halbach magnet array, with \SI{9}{cm} diameter bore and approximately \SI{20}{\centi\metre} long.
This magnet was repurposed from a previous experiment in the lab, and specifications are not available for it.
When combined with the home built coil and holder system described in \autoref{sec:exptsetup-coil}, the FID linewidth produced by the magnet was \SI{12}{\kilo\hertz}, which is relatively broad.

Because it uses permanent magnets, it requires a stable temperature in order to have a stable field.
This was set up using a temperature controlled water bath, which pumped wat at a constant \SI{30}{\celsius} through tubes wrapped around the magnet.
On top of this, the magnet was wrapped in a layer of foam, and a layer of mylar sheeting to insulate it from the room.
To decrease the effect of electrical noise on the measurements, the magnet assembly was also surrounded by a thin coppper mesh blanket.

The second magnet system used was the NMR MOLE, previously developed by Manz et al\cite{ManzmobileonesidedNMR2006}.
The NMR MOLE operates at a field of \SI{0.1}{T} and is a single sided device, which produces a `sweet spot' in the region above the magnet.
The sweet spot is where the combination of magnetic field and RF produced by the coil on the surface are able to create resonance, which defines where signal comes from.
In this case, the sweet spot was a pair of regions marking out a \SI{3}{cm} circle over the RF coil (TODO see figure? honours project).
While earlier experiments attempted to position the tube containing blood over the sweet spot, in later experiments, it was decided that it was simpler and more reliable to place the blood bag on the top of the MOLE.
As with the Halbach array, the MOLE is also sensitive to temperature, so an electronic temperature controller and wire heater were used to keep it stable at \SI{29}{\celsius}.

The design of the magnet array in the MOLE creates strong magnetic field gradients across the sweet spot.
This creates a wide range of resonance frequencies, and means that it was not possible to measure an FID from the system.
It was also found that this limited the possible range of CPMG echo times, as echo times longer than \SI{1}{ms} were caused excessive signal attenuation.
This limited experiments with this system to short echo times.

The third magnet system used was a Magritek Spinsolve, which is another permanent magnet based NMR system operating at \SI{1}{T}.
The Spinsolve is designed for chemical spectroscopy, so has a very homogeneous field, and can produce a linewidth of less than \SI{0.1}{Hz}.
It also uses standard 5 mm NMR sample tubes, so it could not be used inline in the flow circuit like the other two systems.
Because of this, experiments on the Spinsolve required withdrawing a small amount of blood and transferring it into an NMR tube.
This meant that that there were typically delays between removing the blood from the circuit and taking measurements on it, which may cause changes in the state of the blood.

Both the Halbach and MOLE systems were controlled on computers running Magritek Prospa, and used Magritek Kea 2 spectrometers to run the NMR experiments.
The Spinsolve was also controlled from a computer with a newer version of Prospa.
CPMG experiments were run using the default CPMG macros included in Prospa, with batch scripts set up to automate taking measurements at multiple echo times.

\section{Continuous flow setup}
\label{sec:exptsetup-contflow}

To be able to get better data on how \Ttwo is affected by \SOtwo at different fields, we decided to move to a continuous flow method, where the \SOtwo was slowly ramped from low to high oxygenation, while the \Ttwo was constantly measured.
This required a number of changes to the stopped flow setup, including the use of the baby-MRI magnet, and the development of a system for continuours \SOtwo tracking.


\begin{itemize}
\item Flow Circuit changes
\item different magnet
\item sO2 measurement - optical sensor, calibration
\item Flow stability
\end{itemize}

\subsection{Flow Circuit Changes}

\subsection{Coil Assemblies and Electronics}
\label{sec:exptsetup-coil}
A new coil holder, coil assembly and tuning and matching circuit was designed to fit into the baby-MRI system and also used in the Halbach magnet array experiments in  \autoref{ch:stoppedflow}.
It was designed with exchangeable coils and tuning and matching capacitors so that the probe could be used at multiple fields / frequencies.
It consists of a probe (outer piece), which holds the coil assembly and the tuning and matching circuit inside the bore of the magnet (which had the same diameter in the two magnets).

The coils are \SI{2}{cm} long, and have a diameter of \SI{1}{cm}, so that the 1/4'' tube can be passed through the coil,
Different numbers of turns were used for the three different coils, to be able to use the coils at different frequencies (More turns causes more inductance, and a lower resonant frequency.)
In the initial design of the coil assembly, the coil was constructed from \SI{0.67}{mm} Copper wire wrapped around a rolled up acetate cylinder.
A second version of the coil assemblies used 3D printed forms made from ABS plastic (to decrease unwanted signal), with the same wire and number of turns.

TODO:picture of the coils
TODO:picture of the T\&M circuit

The tuning and matching circuit was also designed and manufactured for this probe.
It allows for the capacitors to be exchanged, using small PCBs with different capacitors attached.
While high frequencies
The different capacitors allowed the probe to be tuned and matched at frequencies ranging from \SIrange{2}{60}{\mega\hertz}.
As the probe is located inside the magnet, the capacitors are all of a non-magnetic type (AVX Hi-Q or Cornell Dubilier CDE MCM series) and are rated for high voltage (\textgreater\SI{500}{V}.)
The tuning and matching circuit also contains variable capacitors which are used for fine tuning the match frequency.

S\textsubscript{11} simulations were run in Qucs Spice to find the best values of C\textsubscript{T} and C\textsubscript{M} at a range of frequencies.

\chapter{Stopped flow \SOtwo measurements}\label{ch:stoppedflow}

As a first step in this research, experiments on blood using a stopped flow setup were completed, using a similar protocol and design used in previous work by the group.
These experiments used a flow circuit and 3 different magnet systems with different fields to observe changes in \Ttwo at a number of levels of oxygenation.

Results from these experiments were presented in an oral presentation at the 14\textsuperscript{th} International Congress on Magnetic Resonance Microscopy.

\section{Experimental Setup}

\subsection{Flow circuit}
The blood used in these experiments was contained in a flow circuit similar to what is used for Cardio-Pulmonary Bypass (CPB) surgery.
A schematic of the flow setup is shown in TODO drawing.
The main components are the roller pump and the oxygenation membrane, with the blood collection bag acting as a reservoir.
These are all joined by 1/4'' medical grade PVC tubing.
The total volume of this circuit was typically \SI{60}{\milli\litre}, with the blood bag containing 450 ml TODO check logbook.
The roller pump is used to generate flow around the circuit, by rotating the pump head to drive 2 rollers that squeeze the walls of the tube and force the blood along.
The pump is also designed for use in CPB surgery, and is a Stockert SIII, with a flow rate controllable from \SIrange{0.01}{6}{L/m} when used with the 1/4'' tube. TODO equip?.
These versions of the flow circuit included a bypass section controlled by clamping the tubes so that the blood flow could be directed to around the reservoir, allowing for faster changes in the oxygenation due to a lower effective volume.


\subsection{Oxygenator and Gas Mixer}
A hollow-fibre membrane oxygenator was used to control the oxygenation of the blood in the circuit.
This was a Medtronic Affinity Pixie oxygenator, designed for use in paediatric CPB procedures.
It allows for gas exchange and temperature control of the blood flowing through, with the oxygenation controlled by the mix of gases going into the oxygenator.
This paediatric model was chosen to minimize the required priming volume, and it still provided more than enough capacity to oxygenate the blood at the flow rates used.
The temperature of the blood was regulated by flowing water from a temperature controlled waterbath through the oxygenator, which has a separate path built into it for this purpose.
The water bath temperature was set to \SI{35}{\degreeCelsius}, although this corresponded with a blood temperature at the outlet of \SI{34}{\degreeCelsius} at maximum flow rates (with lower temperatures at lower flow rates.)

The gas mix was set using a Dansensor MapMix 3 gas mixer, which allowed control of the different proportions of N\textsubscript{2}, O\textsubscript{2} and CO\textsubscript{2} flowing through the oxygenator.
The Oxygen fraction was typically varied from 0\% up to 21\%, the value of atmospheric air.
Nitrogen was used as a non-Oxygen containing gas, and between 2.5-5\% CO\textsubscript{2} was also added to ensure the pH of the blood sample remained stable over the course of the experiment (as these are linked by the bicarbonate buffer system in blood).
Gases for all experiments were used as supplied from BOC and were instrument grade or higher.
The output of the gas mixer flowed into a pressurised buffer tank before flowing into a Dansensor MapCheck Provectus gas analyser, which allowed us to check the fraction of the three gases which flowed into the oxygenation membrane.
This gas analyser also provided some flow regulation, although this was supplemented with an adjustable needle valve.
Typically, the gas flow rate was set to \SIrange{1}{2}{\liter\per\minute} to avoid creating bubbles in the oxygenator.

\subsection{Blood collection}
Samples of whole blood were collected by venipuncture from healthy volunteers (taking place at Otago medical school).
This procedure was approved by the central health and disability ethics committee, and all volunteers provided written, informed consent.
Approximately \SI{450}{ml} of blood was collected into a blood bag containing \SI{66.5}{\milli\litre} CPD anticoagulant/preservative solution (Haemonetics Leukotrap WB system).
CPD solution is rated for storage of red blood cells for up to at least 3 weeks of storage\cite{Hessupdatesolutionsred2006}.
Blood samples were held at room temperature following collection, before undergoing leukoreduction filtering to remove white blood cells.
As it has been shown that the presence of white blood cells can adversely affect the condition of red blood cells in storage\cite{Hessupdatesolutionsred2006}.
Samples were then moved to a refrigerator and stored at \SI{4}{\degreeCelsius} until required for experiments (up to 30 TODO check! days.)

\subsection{NMR setup}
For these experiments, 3 different permanent magnet systems were used to obtain data at 3 different field strengths.
The first system was a 12 MHz (\SI{0.3}{T}) Halbach magnet array, with \SI{9}{cm} diameter bore and approximately \SI{20}{\centi\metre} long TODO check!.
This magnet was repurposed from a previous experiment in the lab, and specifications are not available for it.
When combined with the coil and holder system described below, the FID linewidth produced by the magnet was \SI{12}{\kilo\hertz}, which is relatively broad.

Because it uses permanent magnets, it requires a stable temperature in order to have a stable field.
This was set up using a temperature controlled water bath, which pumped wat at a constant \SI{30}{\celsius} through tubes wrapped around the magnet.
On top of this, the magnet was wrapped in a layer of foam, and a layer of mylar sheeting to insulate it from the room.
To decrease the effect of electrical noise on the measurements, the magnet assembly was also surrounded by a thin coppper mesh blanket.

The second magnet system used was the NMR MOLE, previously developed by Manz et al\cite{ManzmobileonesidedNMR2006}.
The NMR MOLE operates at a field of \SI{0.1}{T} and is a single sided device, which produces a `sweet spot' in the region above the magnet.
The sweet spot is where the combination of magnetic field and RF produced by the coil on the surface are able to create resonance, and defines where signal comes from.
In this case, the sweet spot was a pair of regions marking out a \SI{3}{cm} circle over the RF coil (TODO see figure? honours project).
While earlier experiments attempted to position the tube containing blood over the sweet spot, in later experiments, it was decided that it was simpler and more reliable to place the blood bag on the top of the MOLE.
As with the Halbach array, the MOLE is also sensitive to temperature, so an electronic temperature controller and wire heater were used to keep it stable at \SI{29}{\celsius}.

The design of the magnet array in the MOLE creates strong magnetic field gradients across the sweet spot.
This creates a wide range of resonance frequencies, and means that it was not possible to measure an FID from the system.
It was also found that this limited the possible range of CPMG echo times, as echo times longer than \SI{1}{ms} were found to cause excessive signal attenuation.
This limited experiments with this system to short echo times.

The third magnet system used was a Magritek Spinsolve, which is another permanent magnet based NMR system operating at \SI{1}{T}.
The Spinsolve has a very homogeneous field, and can produce a linewidth of \SI{0.1}{Hz}.
This system is designed for chemical spectroscopy, and uses standard 5 mm NMR sample tubes.
Because of this, experiments on the Spinsolve required withdrawing a small amount of blood and transferring it into an NMR tube, rather than using blood in the flow circuit.
This meant that that there were typically delays between removing the blood from the circuit and taking measurements on it, which may cause changes in the state of the blood.

Both the Halbach and MOLE systems were controlled on computers running Magritek Prospa, and used Magritek Kea 2 spectrometers to run the NMR experiments.
The Spinsolve was also controlled from a computer with a newer version of Prospa.
CPMG experiments were run using the default CPMG macros included in Prospa, with batch scripts set up to automate making measurements at multiple echo times.

\subsection{Coil Assemblies and Electronics}
A new coil holder, coil assembly and tuning and matching circuit was designed to fit into the baby-MRI system and the Halbach magnet array.
It was designed with exchangeable coils and tuning and matching capacitors so that the probe could be used at multiple fields / frequencies.
It consists of a probe (outer piece), which holds the coil assembly and the tuning and matching circuit inside the bore of the magnet (which had the same diameter in the two magnets).

The coils are \SI{2}{cm} long, and have a diameter of \SI{1}{cm}, so that the 1/4'' tube can be passed through the coil,
Different numbers of turns were used for the three different coils, to be able to use the coils at different frequencies (More turns causes more inductance, and a lower resonant frequency.)
In the initial design of the coil assembly, the coil was constructed from \SI{0.67}{mm} Copper wire wrapped around a rolled up acetate cylinder.
A second version of the coil assemblies used 3D printed forms made from ABS plastic (to decrease unwanted signal), with the same wire and number of turns.

TODO:picture of the coils
TODO:picture of the T\&M circuit

The tuning and matching circuit was also designed and manufactured for this probe.
It allows for the capacitors to be exchanged, using PCBs with different capacitors attached.
The different capacitors allowed the probe to be tuned and matched at frequencies ranging from \SIrange{2}{60}{\mega\hertz}.
As the probe is located inside the magnet, the capacitors are all of a non-magnetic type and are rated for high voltage > \SI{1000}{V}.
The tuning and matching circuit also contains variable capacitors which are used for fine tuning the match frequency.

\section{Experimental Protocol}


\subsection{Data Processing}
Data was extracted and processed using Python, in JuPyter notebooks.
Each echo train is phased, and each echo is summed to give a single value for the signal at each echo time.
The resulting decay is fit to a monoexponential decay, which results in a single \Ttwo value.




\begin{itemize}
\item Flow Circuit
\item Oxygenator, gas mixer
\item Blood collection and age?
\item NMR setup
\item Coil and electronics
\end{itemize}

\section{Results}
Short section of results presented at ICMRM - 3 sO2 at 3 different fields
\section{Discussions}

\begin{itemize}
\item Hard to set sO2 with gas mixer
\item Effect of blood settling when flow stopped
\item time taken to measure sO2 with iStat
\end{itemize}

%\chapter{Devlopment of a method for continuous \SOtwo measurement}
\label{ch:contsetup}

This chapter is about the system developed to continuously measure the \Ttwo of blood flowing in the circuit as the \SOtwo changes
\section{Experimental setup}

\begin{itemize}
\item Flow Circuit changes
\item different magnet
\item sO2 measurement - optical sensor, calibration
\item Flow stability
\end{itemize}


\section{Results}

Graphs of \Ttwo change at 5 MHz, 10MHz, 14MHz, 20MHz, 40MHz

\section{Discussion}
\begin{itemize}
\item comparisons between different field strengths - agreement with literature
\item mitigation of haemolysis
\item effect of blood velocity on \Ttwo

\end{itemize}

\chapter{Continuous measurements of \Ttwo and \SOtwo}
\label{ch:cont}

This chapter is about all the results from the continuous loop setup
\section{Experimental Method}
\section{Results}

Graphs of \Ttwo change at 5 MHz, 10MHz, 14MHz, 20MHz, 40MHz

\section{Discussion}
\begin{itemize}
\item comparisons between different field strengths - agreement with literature
\item calibration of \Kzero to \SOtwo
\item mitigation of haemolysis
\item effect of blood velocity on \Ttwo

\end{itemize}

\chapter{Models for \Ttwo relaxation enhancement}\label{ch:models}

As discussed in Chapter \ref{ch:background}, researchers have developed a number of models to explain the decrease in \Ttwo of the water protons in blood.
Thulborn originally applied the Luz-Meiboom equation to describe the decrease in \Ttwo and its dependence on the CPMG echo time\cite{ThulbornOxygenationdependencetransverse1982}.
This assumes that the water protons in blood instantaneously exchange between two sites with different Larmor frequencies, with a rate $\frac{1}{\tau_{ex}}$.
While this provides good agreement with experimental data, it does not provide a good description of the underlying physical system - for example, the exchange time in found in previous studies do not agree with measurements of the exchange rate across the red blood cell membrane \cite{MeyerNMRrelaxationrates1995}.
There are also inconsistencies in the values of $\tau_{ex}$ reported in the literature, with values ranging from 0.3 ms to 10.1 ms (see table 2).
Another model has been proposed by Jensen and Chandra, which more accurately describes the dephasing of water protons that diffuse through areas of magnetic field inhomogeneities\cite{JensenNMRrelaxationtissues2000}.
Generally, the two models predict the same behaviour at short echo times (<3 ms) and at long echo times (>20ms)\cite{BrooksT2shorteningweaklymagnetized2001}, but in the intemediate range, the two models give different values for the magnitude of the \Ttwo change.
Additionally, experiments comparing the two models have found that at clinical MRI fields (1.5 T and up), both models provide good agreement with experimental data \cite{StefanovicHumanwholebloodrelaxometry2004,GardenerDependencebloodR22010,GrgacHematocritoxygenationdependence2013}.
Because of this, the Luz-Meiboom equation is typically used when analysing data, as it is simpler than the diffusion model.

As another component of this research, the agreement between these models and data collected at lower fields has been tested.

\section{Theory}

\section{Experimental Setup}

\subsection*{Analysis method}
this
\section{Results}

\section{Discussion}

\chapter{Conclusions and Future Work}\label{ch:conc}

These results confirm that changes in \Ttwo due to blood oxygenation can be detected at low fields in this \textit{in-vitro} system.
The continuous flow experiments showed that this \Ttwo effect follows the trends predicted by the Luz-Meiboom equation, with the increase in \Rtwo and \Kzero being proportional to (1-\SOtwo)\textsuperscript{2} and to \Bzero\textsuperscript{2}.
While some experimental parameters will need to be better controlled in future experiments, the effects of temperature and flow rate do not cause signficant uncertainty in the results.
A decrease in \Ttwo also occured over the course of each experiment, and experiments using UV/Vis spectroscopy suggested this was due to haemolysis occuring in the blood.
By applying the Luz-Meiboom model, measurements using different echo times were used to remove the effect of the background \Ttwo change in the blood samples, allowing for good tracking of the blood \SOtwo.

In a separate series of experiments, the effect of echo time on the \Ttwo decrease was tested.
The Luz-Meiboom equation worked well to describe this effect, although the Jensen and Chandra formula produced better agreement at all field strengths.
This agrees with findings in the literature.

This series of experiments will need to be repeated to become more robust, as there is only a single blood sample used in each experiment for the 5 field strengths.
Comparing the dependence of \Ttwo, \Rtwo and \Kzero on \SOtwo across these samples and field strengths will provide more confidence in the accuracy of these experimental results.
With more experiments, it will also be possible to investigate the effect of haematocrit, which has not been tested in this research.
Experiments with concentrated or diluted blood samples will allow the effect of haematocrit to be isolated from the effect of field strength.

While the results show that changes in blood oxygenation can be detected, there may still be more challenges before this technique can be applied \textit{in-vivo}.
For example these experiments detected the \Ttwo change in blood, but an \textit{in-vivo} system will need to detect the \Ttwo change in blood while in tissue.
Creating a magnet system will also present difficulties, as it can be difficult to optimise parameters such as field strength, field homogeneity and the sweet spot size and depth in a portable NMR system.
It will be important to know what compromises can be made in the design of a real system, while still maintaining good sensitivity to the \Ttwo contrast.

Other types of MRI contrast and techniques also have the potential to be applied in low field portable NMR systems.
In high-field systems, techniques such as diffusion-weighted imaging can be used to detect the early signs of damage to the brain.
Recently, the gradients produced by the NMR MOLE were used to provide accurate measurements of the diffusion coefficients in samples of liquid.
The detection of perfusion is also a clinically valuable measurement that could be implemented with a low field system.

These results serves as a foundation for the future development of a new prototype system.

\chapter*{Conference Attendance}
\label{ch:conference}
The work in this thesis was presented at:

\begin{itemize}
\item D.G. Thomas, Y.C. Tzeng, P Galvosas, S Obruchkov, P.D. Teal, Measurement of transverse relaxation properties of whole blood at low magnetic fields. The 14\textsuperscript{th} International Conference on Magnetic Resonance Microscopy, \textit{oral presentation}, Halifax NS Canada, August 2017
\item D.G. Thomas, Y.C. Tzeng, P Galvosas, S Obruchkov, P.D. Teal, Measurement of transverse relaxation properties of whole blood at low magnetic fields. The 11\textsuperscript{th} Australian and New Zealand Society for Magnetic Resonanc Conference, \textit{poster and short oral presentation}, Kingscliff NSW Australia, December 2017

\end{itemize}


%%%%%%%%%%%%%%%%%%%%%%%%%%%%%%%%%%%%%%%%%%%%%%%%%%%%%%%

% and of course book style knows about backmatter
% \backmatter caused problems with appendices :-(
% and of course report style doesn't
%%%%%%%%%%%%%%%%%%%%%%%%%%%%%%%%%%%%%%%%%%%%%%%%%%%%%%%


\bibliographystyle{ieeetr}
%\bibliographystyle{plain}
\bibliography{mscthesis}


\end{document}
